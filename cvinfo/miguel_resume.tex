%%%%%%%%%%%%%%%%%%%%%%%%%%%%%%%%%%%%%%%%%%%%%%%%%%%%%%%%%%%%%%%%%%
%% 
%% Miguel Cabrera's resume
%%   - based off work by Michael DeCorte and Yisong Yue
%%
%%%%%%%%%%%%%%%%%%%%%%%%%%%%%%%%%%%%%%%%%%%%%%%%%%%%%%%%%%%%%%%%%%%



%%
%% The following code sets up the document formatting
%%

%this assumes that res_yy.sty is in some path
\documentstyle[hyperref, margin, line,textcomp]{res_yy}


%\hypersetup{backref,pdfpagemode=Full,colorlinks=true,backref}

\addtolength{\oddsidemargin}{-0.45in}
\addtolength{\voffset}{-0.30in}
\addtolength{\textwidth}{1.00in} \addtolength{\textheight}{1.50in}

\renewcommand{\namefont}{\LARGE\emph}



%%
%% The following code defines some macros for terms which have raised font
%% (ie 4\fourth would result 4th with the 'th' raised (superscripted)
%%

\def\Cplusplus{{\rm C\raise.7ex\hbox{\scriptsize ++}}}
\def\CSharp{{\rm C\raise.7ex\hbox{\scriptsize \#}}}
% 'st' 'nd' 'rd' 'th' superscripts for numbers
\def\first{{\raise.5ex\hbox{\small st}}}
\def\second{{\raise.5ex\hbox{\small nd}}}
\def\third{{\raise.5ex\hbox{\small rd}}}
\def\fourth{{\raise.5ex\hbox{\small th}}}



%%
%% starting the actual document
%%

\begin{document}

%the name in big fonts at the top of resume
%this is left aligned
\name{Miguel Fernando Cabrera Granados}

%this is right aligned
\address {  \begin{tabular}{lllll}
  Cll 50 \# 71 - 80 Apt 1015  & \ \ & +57 4 260 82 26 \\
  Edf. El Caribe -Sector Estadio  & \ \ & +57 300 492 8461   \\
  Medell\'{i}n - Antioquia & \ \ & miguelcabrera@acm.org \\  
  Colombia  & \ \ & http://mfcabrera.com \\
  \end{tabular}
}

\begin{resume}

%%
%% This section of code is inelegant, but I'm too lazy to fix it
%%

%% Correct this stating that IR is also important for me, not only for
% application.

\section{\textsc{Objective}}
%To work in a challenging environment and build world-class software solutions.
My research interests lie primarily in Machine Learning and its
applications.
%  specially but  not limited, to Information Retrieval and Filtering, Intelligent Information Systems, Recommender Systems, Desktop and Enterprise Search.

%\section{\textsc{Research\\ Interest}}
%My research interests lie primarily in Machine Learning and its application  specially but  not limited to Information Retrieval and Filtering, Intelligent Information Systems, Recommender Systems, Desktop and Enterprise Search.


\section{\textsc{Education}}
\textbf{National University of Colombia - Medell\'{i}n} Campus \hfill 2003 - Present \\
{Systems and Informatic Engineer} \hfill 


\section{\textsc{Course\\ Work}}
  \begin{tabular}{lllll}
Information Networks   & \ \ &  Operation Research   & \ \ &
Multiagent Systems  \\ 
Digital Image Processing     & \ \ & Complex Systems  & \ \ &
Declarative Languages  \\
Software Engineering   & \ \ & Algorithms          & \ \ & Artificial Intelligence     \\
Operating Systems      & \ \ & Databases I and II          & \ \ & Computer Architecture \\
Numerical Methods      & \ \ & Macroeconomics       & \ \ & Differential Equations      \\
Basic Probability      & \ \ & Microeconomics      & \ \ & Linear Algebra and Linear Systems      \\
Advanced Calculus      & \ \ & Linear Algebra    & \ \ & Computational Intelligence   \\
\end{tabular}
\newline

\section{\textsc{Awards}}
\textbf{Andr\'{e}s Bello Medal} \hfill          Bogota - Colombia - 2000 \\
Given by the Education Ministry to the best scores in the ICFES National pre-university exam \hfill \\

\section{\textsc{Languages}}
\textbf{Spanish}\   - \  Mother tongue \\
\textbf{English}\ \  - \  

%%
%% the meat of the resume starts now
%%

\begin{formatb}
  \title{l}\employer{l}\\
 \location{r}\dates{r}\\
  \body\\
\end{formatb}

\section{\textsc{Work\\ Experience}}

\employer{\textbf{March 2007 - Present }}
\title{\textbf{Web Developer} (Part time)}
\location{School of Systems, National University of Colombia} 
\dates{Medell\'{i}n, Colombia }
\begin{position}
Provided technical Web expertise that helped to enhance Systems School's corporate web site, 
as defined in the University's Web Publishing Policy. Maintained and configured the School's web servers. Developed and maintained the staff directory, student and staff intranets which retrieved information from MySQL databases using PHP programs. Implemented a MVC Based model.
%\textbf{Supervisor}:  Dr. Fernando Arnango Isaza,Ms.c, Ph.d.\\
%\textbf{Telephone}: (+57 4) 425 5362 
\end{position}

\employer{\textbf{January 2007 - May 2007}}
\title{\textbf{Research Assistant}}
\location{School of Geological Sciences, National University of Colombia}
\dates{Medell\'{i}n, Colombia}
\begin{position}
I Worked in the set-up, running and modification of the NOAA's WaveWatch III Model under a Linux based Cluster
for high performance wave height prediction on the Caribbean Sea.
%\textbf{Supervisor}:  Dr. Andr\'{e}s Osorio,Ms.c, Ph.d.\\
%\textbf{Telephone}: (+57 4) 425 5223
\end{position}


\employer{\textbf{January 2006 - December 2006}}
\title{\textbf{Open Source Engineer} (Part Time)}
\location{{Soluciones Kazak}}
\dates{Medell\'{i}n, Colombia}
\begin{position}
My team developed custom  Linux distribution for our clients
special needs. I also worked on  the set-up of many enterprise
servers ranging from small ISP to medium size companies. http://www.kazak.com.co
%I worked with Ubuntu, CentOS Linux distribution and NetBSD. 
%Programming languages used: Perl and Ruby.
%\textbf{Supervisor}:  Ing. Gustavo Gonzales.\\
%\textbf{Telephone}:  (+57 2) 318 06 23 
\end{position}


\employer{\textbf{February 2005 - December 2005}}
\title{\textbf{Software Developer}}
\location{{Interplanet}}
\dates{Envigado, Colombia}
\begin{position}
Interplanet is an  ISP located in the city of Envigado, metropolitan area of Medell\'{i}n.
My work consisted mostly  in internal development in the form of
Middleware applications  around the company's information systems.
We performed modification to the Radius protocol software used inside the company.
I also worked as an outsourced software developer maintaining a Web
application for one of our clients.  I used Perl and C for internal
development and  system administration tasks and Java JSP/Servlet
framework for Web
Application development.
%\textbf{Supervisor}:  Ing. Jaime Roldan.\\
%\textbf{Telephone}:  (+57 4) 331 0623 
\end{position}
\newline
\newline
\newline
\newline
\newline
\newline
%% \newline
%% \newline
%% \newline

\employer{\textbf{September 2003 - December 2004}}
\title{\textbf{Server Administrator - Technical Support}}
\location{Systems Laboratory, National University of Colombia}
\dates{Medell\'{i}n, Colombia}
\begin{position}
I administered two Linux servers, which hosted the Systems School's
cororate and staff pages.I provided support to technical problems 
to computer rooms users and coordinated the computer room maintenance.
%\textbf{Supervisor}:  John Jairo Tabares.\\
%\textbf{Telephone}:  (+57 4) 425 5312 
\end{position}
%\newline


%% \employer{\textbf{Interspersed 2006 and 2007}}
%% \title{\textbf{Web Developer}}
%% \location{School of Systems and Informatics, NUC}
%% \dates{Medell\'{i}n, Colombia}
%% \begin{position}
%% 2007 - 2006 Version of EITI Conference Webpage (Ecuentro De Investigaci\'{o}n Sobre Tecnologías de Informaci\'{o}n Aplicadas a las Soluci\'{o}n de Problemas).
%% \end{position}



%%
%% We use the same formatting for projects as for work experience
%% Shown below is the formatting used previously
%%
%%  \begin{formatb}
%%    \employer{l}\title{r}\\
%%    \location{l}\dates{r}\\
%%    \body\\
%%  \end{formatb}
%%
%% 
%%  Note that \location is now being used for non-location information
%%
 \begin{formatb}
   \employer{l}\dates{r}\\
   \body\\
 \end{formatb}
%% 
%% \section{\textsc{Publications}}
%% 
%% \employer{\textbf{A Support Vector Method for Optimizing Average Precision}}
%% \dates{SIGIR 2007}
%% \begin{position}
%% Authors: Yisong Yue, Thomas Finley, Filip Radlinski, Thorsten Joachims
%% \end{position}


%%
%% This section could also use more formatting, but looks ok, as is
%%

\section{\textsc{Technical\\ Skills}}

\emph{Programming Languages}: C/\Cplusplus,\ \CSharp,\ PHP,Java, Ruby, SQL, Perl, JavaScript.

\emph{Libraries and Tools}: Emacs, \LaTeX, GIMP, MatLab, Eclipse,GTKSharp, GCC, GDB, Diff, Patch, SVN, CVS, JSP/Servlets, Rails.

\emph{Operating Systems}: Advanced user and administration skills in Linux (Ubuntu,SuSE,Debian,CentOS). Advanced Windows user.

\emph{Others}: XML,\ (X)HTML, CSS, MySQL, PosgreSQL, Oracle, UML, LyX, Apache.


\section{\textsc{Other\\  Projects}}

\employer{\textbf{Parallel Transductive SVMs for Document Classification}}
\dates{August 2007 - Present}
\begin{position}
We developed a paralell implementation of a transductive
linear support vector machine. The main Idea of the work was
to find out whether the parallelization of such algorithm using a cascade
setup have any influence on its performance. This was my thesis project.\\
\textbf{Advisor}:  Dr. Jairo Espinosa, Msc,Ph.d.
\end{position}

\employer{\textbf{Web based Repository of Range Images}}
\dates{March-July 2006}
\begin{position}
 In this project I developed a Web based repository for Range Images, the goal of the
 project was to create a centralized repository for all the range
 images generated from research projects around world in the field of
 object reconstruction and surface fitting. The application was
 developed using Ruby on Rails with a MySQL database.
% \textbf{Supervisor}: Dr. John William Branch Msc,Ph.d.
\end{position}

\employer{\textbf{Beagle CHM Filter}}
\dates{May-July 2005}
\begin{position}
Beagle a Desktop search tool  for Linux similar to
Google\texttrademark  \  Desktop
Search. I developed a filter for CHM (Windows' Help format) indexing.
http://www.beagle-project.org
\end{position}
%%
%% Note that we're redefining the formatting
%% We only have one row of information now, instead of two
%%

\section{\textsc{Extracurricular\\ Activities}}

\begin{formatb}
  \employer{l}
  \dates{r}\\
   
  \body\\
\end{formatb}

\employer{\textbf{UNALIX}}
\dates{June 2003 - January 2006}
\begin{position}
UNALIX is the University's Linux and Open Source User Group, I was one
of the founder member  and general coordinator until January of
2006. I am still an active member. UNALIX is both an user group and academic group where we
discuss the technology and the applicability of Open Source Software
in the University's environment. http://www.unalix.org
\end{position}


%% \employer{\textbf{ParqueSoft Medell\'{i}n}}
%% \dates{January 2007 -  June 2007}
%% \begin{position}
%% I am one of the founder of ParqueSoft Medell\'{i}n, A local branch of
%% ParqueSoft, a technology cluster that fosters entrepreneurship which have its 
%% main headquarters in the city of Cali.  I was part of the local directive board
%% as a representative of the entrepreneurs.
%% \end{position}

\section{\textsc{Technical \\ Talks}}

\begin{formatb}
  \employer{l}\\
  \dates{r}\location{l}\\
   \body
\end{formatb}

\employer{\textbf{Web Development for Programmers: Introduction to Ruby on Rails}}
\dates{October 2006}
\location {University of The District - Bogota, Colombia}
\begin{position}
V District University Linux Week (SLUD-V).
\end{position}

\employer{\textbf{Enterprise Open Source Massive Migration.}}
\dates{October 2006}
\location {University of The District - Bogota, Colombia}
\begin{position}
District  University Linux Week (SLUD-V).
\end{position}

\employer{\textbf{Introduction to Mono Development Environment}}
\dates{September 2005}
\location {CECAR - Sincelejo, Colombia}
\begin{position}
Software Freedom Day Celebration.
\end{position}
%% \newline



\section{\textsc{Organization}}
\employer{\textbf{}}
\dates{}
\textbf{III Free Software International Congress}, Organizing member. Medell\'{i}n - Colombia. April 2004.
%\begin{position}
%\end{position}
\newline     
\newline
\newline     
\newline
\newline     
\newline          
\newline     
\newline     
\newline     
\newline     

\section{\textsc{Professional\\ Memberships}}

\begin{formatb}
  \employer{l}\dates{r}\\
  \body\\
\end{formatb}
\employer{}
   {\textbf{UNALIX} - University's Linux and Open Source User Group,
     June 2003 - Present \\ \\
    \textbf{Association of Computing and Machinery (ACM) Student
      Member}, May 2005 - Present \\ \\
    \textbf{Ubuntu Colombia Team}, June 2006 - Present 
     }

%\dates{}
%\begin{position}
%\end{position}


%Redefining the format
\begin{formatb}
  \employer{l}\title{r}\\
  \location{l}\dates{r}\\
 \body\\
\end{formatb}

\section{\textsc{Teaching\\ Experience}}
\employer{\textbf{Extension Courses Instructor}}
\title{\textbf{June to July 2005}}
\location{Systems Laboratory,National University of Colombia} 
\dates{Medell\'{i}n, Colombia }
\begin{position}
Introduction to Web Development with PHP and MySQL extension course.
\end{position}


%%
%% Nothing special here, just a normal table
%%

%\section{\textsc{Course Work}}
%  \begin{tabular}{lllll}
%  Information Networks   & \ \ & Machine Learning    & \ \ & Theory of Computation \\ 
%  Computer Graphics      & \ \ & Machine Vision      & \ \ & Programming Languages \\
%  Software Engineering   & \ \ & Algorithms          & \ \ & Artificial Intelligence     \\
%  Operating Systems      & \ \ & Databases           & \ \ & Computer Architecture \\
%  Numerical Methods      & \ \ & Graph Theory        & \ \ & Differential Equations      \\
%  Probability Theory     & \ \ & Number Theory       & \ \ & Differential Geometry       \\
%  Advanced Calculus      & \ \ & Abstract Algebra    & \ \ & Advanced Combinatorics   \\
%  \end{tabular}

\section{\textsc{Personal\\ Interests}}
\employer{}
\title{}
\location{} 
\dates{}
Brazilian Jiu-Jitsu, Judo, Film Critic, Web 2.0 Technologies and trends, Open Source Software, Hiking.

\section{\textsc{Referees}}

Ing. Jairo Jos\'{e} Espinosa Msc,Phd.\\
Associate Professor \\
School of Electrical Engineering \\
Faculty of Mines \\
National University of Colombia - Medell\'{i}n Campus\\
Phone: +57 4 4255284 \\
E-mail: jairo.espinosa@ieee.org\\
\\  
Ing. John William Branch Msc,Phd.\\
Associate Professor \\
School of Systems  \\
Faculty of Mines \\
National University of Colombia - Medell\'{i}n Campus\\
Phone: +57 4 4255375 \\
E-mail: jwbranch@unalmed.edu.co\\
\\  
Ing. Andr\'{e}s Osorio Arias Msc,Phd.\\
Associate Professor \\
School of Geological and Environmental Sciences \\
Faculty of Mines \\
National University of Colombia - Medell\'{i}n Campus\\
Phone: +57 4 4255220 \\
E-mail: afosorioar@unal.edu.co\\
\\
Ing. Fernando Arango Isaza Msc,Phd.\\
Associate Professor \\
School of Systems  \\
Faculty of Mines \\
National University of Colombia - Medell\'{i}n Campus\\
Phone: +57 4 4255362 \\
E-mail: farango@unal.edu.co\\

%% \\  
%Ing. Esteban Salazar Guzman\\
%Software Analyst \\
%Susalud S.A. \\
%Phone:+57  4 4938320 \\
%Cell: +57 3003943091 \\
%E-mail: esalaza@gmail.com\\


%% Ing. Rafael Ignacio Larios\\
%% Software Analyst \\
%% Grupo Bancolombia S.A. \\
%% Phone: +57 4  \\
%% E-mail: esalaza@gmail.com\\
%% \\  


\end{resume}
\end{document}
