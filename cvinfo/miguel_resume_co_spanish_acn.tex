%%%%%%%%%%%%%%%%%%%%%%%%%%%%%%%%%%%%%%%%%%%%%%%%%%%%%%%%%%%%%%%%%%
%% 
%% Miguel Cabrera's resume
%%   - based off work by Michael DeCorte and Yisong Yue
%%
%%%%%%%%%%%%%%%%%%%%%%%%%%%%%%%%%%%%%%%%%%%%%%%%%%%%%%%%%%%%%%%%%%%



%%
%% The following code sets up the document formatting
%%

%this assumes that res_yy.sty is in some path
\documentstyle[hyperref, margin, line,textcomp]{res_yy}
%\hypersetup{backref,pdfpagemode=Full,colorlinks=true,linkcolor=blue}

\addtolength{\oddsidemargin}{-0.45in}
\addtolength{\voffset}{-0.30in}
\addtolength{\textwidth}{1.00in} \addtolength{\textheight}{1.50in}

\renewcommand{\namefont}{\LARGE\emph}



%%
%% The following code defines some macros for terms which have raised font
%% (ie 4\fourth would result 4th with the 'th' raised (superscripted)
%%

\def\Cplusplus{{\rm C\raise.7ex\hbox{\scriptsize ++}}}
\def\CSharp{{\rm C\raise.7ex\hbox{\scriptsize \#}}}
\def\GTKSharp{{\rm GTK\raise.7ex\hbox{\scriptsize \#}}}
% 'st' 'nd' 'rd' 'th' superscripts for numbers
\def\first{{\raise.5ex\hbox{\small st}}}
\def\second{{\raise.5ex\hbox{\small nd}}}
\def\third{{\raise.5ex\hbox{\small rd}}}
\def\fourth{{\raise.5ex\hbox{\small th}}}



%%
%% starting the actual document
%%

\begin{document}

%the name in big fonts at the top of resume
%this is left aligned
\name{Miguel Fernando Cabrera Granados}

%this is right aligned
\address {  \begin{tabular}{lllll}
  Cl 32EE \# 80 - 60 APT 402  & \ \ & +57 4 12 28 54 \\
  Edf. Samaria - Barrio Laureles  & \ \ & +57 300 492 8461   \\
  Medell\'{i}n - Antioquia & \ \ & mfcabrera@gmail.com \\  
  Colombia  & \ \ & http://mfcabrera.com \\
  \end{tabular}
}

\begin{resume}

%%
%% This section of code is inelegant, but I'm too lazy to fix it
%%
%% Correct this stating that IR is also important for me, not only for
% application.

\section{\textsc{Perfil}}
Ingeniero de Sistemas e Inform\'{a}tica bilingue con amplia
experiencia en Linux y Sistemas Open Source.  Conocimiento de m\'{u}ltiples
lenguajes y paradigmas de programaci\'{o}n, gran capacidad de
aprendizaje de nuevas tecnolog\'{i}as y proficiencia en la gesti\'{o}n
resoluci\'{o}n de problemas.  Apasionado por las tecnolog\'{i}as, la
ingenier\'{i}a del software y la inteligencia artificial. Se ha desempeñado como desarrollador de aplicaciones
Web, Administrador de Sistemas tipo UNIX y Analista de Integraci\'{o}n,

%\section{\textsc{Objetivo}}
%Trabajar en un entorno estimulante y crear soluciones
%inform\'{a}ticas de clase mundial.
%To work in a challenging environment and build world-class software solutions.
%My research interests lie primarily in Machine Learning and its application  specially but  not limited to Information Retrieval and Filtering, Intelligent Information Systems, Recommender Systems, Desktop and Enterprise Search.

%\section{\textsc{Research\\ Interest}}
%My research interests lie primarily in Machine Learning and its application  specially but  not limited to Information Retrieval and Filtering, Intelligent Information Systems, Recommender Systems, Desktop and Enterprise Search.


\section{\textsc{Educaci\'{o}n}}
\textbf{Universidad Nacional de Colombia  - Sede  Medell\'{i}n} \hfill 2003 - 2007 \\
{Ingeniero de Sistemas e Inform\'{a}tica } \hfill

%% \section{\textsc{Cursos}}
%%   \begin{tabular}{lllll}
%% Redes y Telecomunicaciones & \ \ &  Investigacion de Operaciones   & \ \ &
%% Sistemas Multiagentes  \\ 
%% Procesamiento Digital de Im\'{a}genes   & \ \ & Sistemas Complejos  &\ \ & Languages Declarativos  \\
%% Ingenier\'{i}a de Software   & \ \ & Algoritmos          & \ \ & Inteligencia Artificial     \\
%% Sistemas Operativos     & \ \ & Bases de Datos I y II   & \ \ & Arquitecura de Microcomp. \\
%% M\'{e}todos Num\'{e}ricos      & \ \ & Macroeconomia       & \ \ & Differential Equations      \\
%% Estad\'{i}stica I y II   & \ \ & Microeconomia      & \ \ & Sistemas Din\'{a}micos    \\
%% Calculo Avanzado     & \ \ & Algebra Lineal    & \ \ & Inteligencia Computacional   \\
%% \end{tabular}
%% \newline

\section{\textsc{Reconocimientos}}
\textbf{Menci\'{o}n Andr\'{e}s Bello} \hfill  Sincelejo - Sucre 2000  \\
Otorgado por el Ministerio de Eduaci\'{o}n a los mejores puntajes en
las pruebas del ICFES \hfill \\

\section{\textsc{Idiomas}}
\textbf{Espa\~{n}ol}\   - \  Lengua Materna \\
\textbf{Ingl\'{e}s}\ \  -  Fluente (TOEFL 105/120 - Melicet (Michigan) 90/100)



%%
%% the meat of the resume starts now
%%

\begin{formatb}
  \title{l}\employer{l}\\
 \location{r}\dates{r}\\
  \body\\
\end{formatb}

\section{\textsc{\\Experiencia \\ Laboral}}


\employer{\textbf{Febrero 2008 - Presente}}
\title{\textbf{Analista} }
\location{Accenture} 
\dates{Medell\'{i}n, Colombia }
\begin{position}
Miembro del equipo de interfaces pora la nueva plataforma de negociaci\'{o}n  del
Grupo Bancolombia, Murex. Mis responsabilidades pueden ser
resumidas en:
\begin{itemize}
\item {Interactuar con los stakeholders con el objetivo de extraer los requerimientos de las interfaces de Murex con sistemas externos.}
\item {Coordinaci\'{o}n con el grupo de tecnolog\'{i}a del banco para con el objetivo de realizar los cambios tecnol\'{o}gicos necesarios para asegurar el  funcionamiento de Murex, incluyendo la instalaci\'{o}n de Middlewares y sistemas de Datos de Mercado/Transaccionales como EBS y Bloomberg.}
\item {Implementaci\'{o} de Worfkflows de importaci\'{o}n de datos est\'{a}ticos y de operaciones}
\item {Dise\~{o} de las Interfaces y Mapeo de formatos entre sistemas del Banco y Murex}
\item {Conectar los sistemas legacy del banco con Murex usando tecnolog\'{i}as basadas en XML.}
\item {Desarrollo en C/\CSharp usando MXSA para conectar Murex con un sistema de Trading local.}
\end{itemize}

Tecnolog\'{i}as: C,XML, Java, IBM  Websphere MQ, IBM AIX, Linux and Murex MX.3.
\end{position}

\employer{\textbf{Marzo 2007 - Diciembre 2007}}
\title{\textbf{Desarrollador Web} }
\location{Universidad Nacional de Colombia} 
\dates{Medell\'{i}n, Colombia }
\begin{position}
Desarroll\'{e} mejoras en la web corporativa de la escuela de Escuela
de Sistemas e Inform\'{a}tica. Configuraba y  co-administraba los
servidores Web de la escuela. Desarroll\'{e} estrat\'{e}gias para
mejorar la usabiliad, agregu\'{e} m\'{o}dulos faltantes relacionados
con la informaci\'{o}n de los cursos y estudiantes. Se implement\'{o}
un modelo MVC. La aplicaci\'{o}n fue desarrollada usando PHP y
MySQL.
%Adicinalmente trabaj\'{e} en la instalaci\'{o}m, configuraci\'{o}n y ejecuci\'{o}  del modelo WaveWatch III de NOAA bajo un cluster basado en Linux con el objetivo de hacer predicciones del oleaje del mar Caribe con alto desempe\~{n}o.\\
%I Worked in the s
%Provided technical Web expertise that helped to enhance Systems and Informatics School's corporate web site, as defined in the University's Web Publishing Policy. Maintained and configured the School's web servers. Developed and maintained the staff directory, student and staff intranets which retrieved information from MySQL databases using PHP programs. Implemented a MVC Based model.\\
%\textbf{Supervisor}: Dr. Fernando Arnango Isaza. Ms.c, Ph.d.\
%\textbf{Supervisor}: Dr. Andr\'{e}s Osorio. Ms.c, Ph.d.\\
%\textbf{Tel\'{e}fono}: (+57 4) 425 5223
%%\textbf{Tel\'{e}fono}: (+57 4) 425 5362 
\end{position}

%\employer{\textbf{Enero 2007 - Mayo 2007}}
 %\title{\textbf{Asistente de Investigaci\'{o}n}}
 %\location{Escuela de Ciencias Geol\'{o}gicas, Universidad Nacional de Colombia}
 %\dates{Medell\'{i}n, Colombia}
 %\begin{position}
 %Trabaj\'{e} en la instalaci\'{o}m, configuraci\'{o}n y ejecuci\'{o}  del modelo WaveWatch III de NOAA bajo un cluster basado en Linux con el objetivo de hacer predicciones del oleaje del mar Caribe con alto desempe\~{n}o.

%% %I Worked in the set-up, running and modification of the NOAA's WaveWatch III Model under a Linux based Cluster
%% %for high performance wave height prediction on the Caribbean Sea.\\
%% \textbf{Supervisor}: Dr. Andr\'{e}s Osorio. Ms.c, Ph.d.\\
%% \textbf{Tel\'{e}fono}: (+57 4) 425 5223
% \end{position}

\employer{\textbf{Enero 2006 - Diciembre 2006}}
\title{\textbf{Open Source Engineer}}
\location{{Soluciones Kazak}}
\dates{Medell\'{i}n, Colombia}
\begin{position}
Mi equipo desarroll\'{o} una distribuci\'{o}n de Linux ajustada a las
necesidades espec\'{i}ficas de un cliente, para posteriormente llevar
a cabo una  migraci\'{o}n de servidores y estaciones de trabajo. En ese tiempo tambi\'{e}n trabaj\'{e} en
la instalaci\'{o}n y configuraci\'{o}n de varios servidores para
distintas empresas.
%My team developed custom  Linux distribution for our clients
%special needs. I also worked on  the set-up of many enterprise
%servers ranging from small companies to medium size ISP. http://www.kazak.com.co\\
%I worked with Ubuntu, CentOS Linux distribution and NetBSD. 
%Programming languages used: Perl and Ruby.
%\textbf{Supervisor}: Ing. Gustavo Gonzales.\\
%\textbf{Tel\'{e}fono}:  (+57 2) 318 06 23 - (+57) 316 326 47 83
\end{position}
 \newline
 \newline
 \newline
 \newline
 \newline
                                   


\employer{\textbf{Febrero 2005 - Diciembre 2005}}
\title{\textbf{Desarrollador de Software}}
\location{{Interplanet}}
\dates{Envigado, Colombia}
\begin{position}
Interplanet es una empresa proveedora de servicios de Internet. Mi trabajo consiti\'{o} la mayor\'{i}a del tiempo en desarrollo de middleware alrededor de los sistemas de informaci\'{o}n que tenia  la compa\~{n}ia.
Trabaj\'{e} los ultimos dos meses como desarrollador manteniendo una aplicaci\'{o}n Web para uno de los clientes de la empresa. Us\'{e} Perl y C para el desarrollo interno y tareas de administraci\'{o}n y JSP/Servlets para el desarrollo Web.\\
%I also worked as an outsourced software developer maintaining a Web application for one of our clients.  I used Perl and C for internal development and  system administration tasks and Java JSP/Servlet framework Web
\textbf{Supervisor}:  Ing. Jaime Roldan.\\
\textbf{Tel\'{e}fono}:  (+57 4) 331 0623 
\end{position}
%% \newline
%% \newline
%% \newline
%% \newline
%% \newline
%% \newline
%% \employer{\textbf{Interspersed 2006 and 2007}}
%% \title{\textbf{Web Developer}}
%% \location{School of Systems and Informatics, NUC}
%% \dates{Medell\'{i}n, Colombia}
%% \begin{position}
%% 2007 - 2006 Version of EITI Conference Webpage (Ecuentro De Investigaci\'{o}n Sobre Tecnolog\'{i}as de Informaci\'{o}n Aplicadas a las Soluci\'{o}n de Problemas).
%% \end{position}



%%
%% We use the same formatting for projects as for work experience
%% Shown below is the formatting used previously
%%
%%  \begin{formatb}
%%    \employer{l}\title{r}\\
%%    \location{l}\dates{r}\\
%%    \body\\
%%  \end{formatb}
%%
%% 
%%  Note that \location is now being used for non-location information
%%
 \begin{formatb}
   \employer{l}\dates{r}\\
   \body\\
 \end{formatb}
%% 
%% \section{\textsc{Publications}}
%% 
%% \employer{\textbf{A Support Vector Method for Optimizing Average Precision}}
%% \dates{SIGIR 2007}
%% \begin{position}
%% Authors: Yisong Yue, Thomas Finley, Filip Radlinski, Thorsten Joachims
%% \end{position}


%%
%% This section could also use more formatting, but looks ok, as is
%%

\section{\textsc{Conocimientos}}

\emph{Tecnolog\'{i}as }: Sistemas Linux y Unix, Integraci\'{o}n de
Aplicaciones Enterprise, Open Source Software, Aprendizaje de
M\'{a}quinas (SVM).


\emph{Lenguajes de Programaci\'{o}n}: Tengo experiencia en
C/\Cplusplus,\ \ PHP y Java. He logrado expericia en
JavaScript, \ \CSharp, Ruby, Pyhton a trav\'{e}s de proyectos
personales y como freelance.

\emph{Librerias y Herramientas}: Emacs, \LaTeX, GIMP, MatLab, Eclipse, \GTKSharp,\ GCC, GDB, Diff, Patch, SVN, CVS, JSP/Servlets, Rails, LibSVM.

\emph{Sistemas Operativos}: Usuario avanzado en Linux  y conocimientos en
administraci\'{o}n de servidores Linux (Debian, Suse, CentOS, Ubuntu). Usuario avanzado de Windows. Usuario de
otros Sistemas tipo UNIX (AIX, NetBSD).

\emph{Otros}: XML,\ (X)HTML, CSS2, MySQL, PosgreSQL, Oracle,
UML, LyX, Apache, WebSphere MQ, Websphere Application
Server. Conocimientos b\'{a}sicos en mercados financieros.


\section{\textsc{Otros\\  Proyectos}}

\employer{\textbf{Parallel Transductive SVMs for Document Classification}}
\dates{Agosto 2007 - Diciembre 2007}
\begin{position}
Desarroll\'{e} una implementaci\'{o}n  paralela en cascada de la version transductiva de una maquina de soporte vectorial
para la clasificaci\'{o}n autom\'{a}tica de documentos. Proyecto de Grado(Escrita en Ingles).\\
\textbf{Director}:  Dr. Jairo Espinosa, Msc,Ph.d.
\end{position}

\employer{\textbf{Repositorio Web para Imagenes de Rango}}
\dates{Marzo -Julio 2006}
\begin{position}
 Se desarroll\'{o} un respositorio web de imagenes de rango generadas por un scanner 3D  usando Ruby on Rails y una base de datos MySQL.
El objetivo del proyecto era crear un repositorio mundial centralizado de estas imagenes. \\
% In this project I developed a Web based repository for Range Images, the goal of the
 %project was to create a centralized repository for all the range
% images generated from research projects around world in the field of
% object reconstruction and surface fitting. The application was
% developed using Ruby on Rails with a MySQL database.\\
 \textbf{Supervisor}: Dr. John William Branch Msc,Ph.d.
\end{position}

\employer{\textbf{Beagle CHM Filter}}
\dates{Mayo - Julio 2005}
\begin{position}
Beagle es una herramienta Open Source para Google\texttrademark  \  Desktop Search. Desarroll\'{e} un filtro que permite
indexar ficheros en formato CHM (Formato de Ayuda de Windows).
%Beagle a Desktop search tool  for Linux similar to
%Google\texttrademark  \  Desktop Search. I developed a filter for CHM (Windows' Help format) indexing.
http://www.beagle-project.org
\end{position}
%%
%% Note that we're redefining the formatting
%% We only have one row of information now, instead of two
%%

\section{\textsc{Actividades \\ Extracurriculares }}

\begin{formatb}
  \employer{l}
  \dates{r}\\
   
  \body\\
\end{formatb}

\employer{\textbf{UNALIX}}
\dates{Junio 2003 - Enero 2006}
\begin{position}
UNALIX es el grupo de usuarios de Linux y Software Libre de la
Universidad Nacional de Colombia - Sede Medell\'{i}n. Soy miembro
fundador y fu\'{i} coordinador general hasta Enero del 2005. http://www.unalix.org.
%UNALIX is the University's Linux and Open Source User Group, I was one
%of the founder member  and general coordinator until Enero of
%2006. I am still an active member. UNALIX is both an user group and academic group where we
%discuss the technology and the applicability of Open Source Software
%in the University's environment. http://www.unalix.org
\end{position}

%% \employer{\textbf{ParqueSoft Medell\'{i}n}}
%% \dates{Enero 2007 -  June 2007}
%% \begin{position}
%% I am one of the founder of ParqueSoft Medell\'{i}n, A local branch of
%% ParqueSoft, a technology cluster that fosters entrepreneurship which have its 
%% main headquarters in the city of Cali.  I was part of the local directive board
%% as a representative of the entrepreneurs.
%% \end{position}

\section{\textsc{Charlas \\ T\'{e}cnicas \\ (Dadas)}}

\begin{formatb}
  \employer{l}\\
  \dates{r}\location{l}\\
   \body
\end{formatb}

\employer{\textbf{Desarrollo Web para Programadores: Introducci\'{o}n a Ruby on Rails}}
\dates{Octubre de 2006}
\location {Universidad Distrital - Bogota, Colombia}
\begin{position}
V Semana Linux Universidad Distrital (SLUD-V).
\end{position}


\employer{\textbf{Enterprise Open Source Massive Migration.}}
\dates{Octubre de 2006}
\location {Universidad Distrital - Bogota, Colombia}
\begin{position}
V Semana Linux Universidad Distrital (SLUD-V).
\end{position}


\employer{\textbf{Introducci\'{o}n a la Plataforma de Desarrollo Mono}}
\dates{Septiembre de 2005}
\location {Corporaci\'{o}n Universitaria del Caribe - CECAR - Sincelejo, Colombia}
\begin{position}
Celebraci\'{o}n del D\'{i}a Mundial del Software Libre.
\end{position}
\newline
\newline
\newline
\newline
\newline

%% \newline



\section{\textsc{Organizaci\'{o}n}}
\employer{\textbf{}}
\dates{}
\textbf{III Congreso Internacional de Software Libre},Miembro del Comite Organizador. Medell\'{i}n - Colombia. April 2004.
%\begin{position}
%\end{position}
\newline     


\section{\textsc{Membres\'{i}as\\ Profesionales}}

\begin{formatb}
  \employer{l}\dates{r}\\
  \body\\
\end{formatb}
\employer{}
   {\textbf{UNALIX} - Grupo Linux Universidad Nacional de Colombia,
     Junio de 2003 - Presente \\ \\
    \textbf{Association of Computing and Machinery (ACM) - Miembro Profesional}, Mayo del  2005 - Presente \\ \\

     }

%\dates{}
%\begin{position}
%\end{position}


%Redefining the format
\begin{formatb}
  \employer{l}\title{r}\\
  \location{l}\dates{r}\\
 \body\\
\end{formatb}

 \section{\textsc{Experiencia \\ Docente}}
 \employer{\textbf{Instructor de Cursos de Extensi\'{o}n}}
 \title{\textbf{Junio y Julio de 2005}}
 \location{Laboratoria de Sistemas, Universidad Nacional de Colombia} 
 \dates{Medell\'{i}n, Colombia }
 \begin{position}
 \emph{Introducci\'{o}n a C} y \emph{Desarrollo de Aplicaciones Web
 con PHP y MySQL}
 \end{position}


%%
%% Nothing special here, just a normal table
%%

%\section{\textsc{Course Work}}
%  \begin{tabular}{lllll}
%  Information Networks   & \ \ & Machine Learning    & \ \ & Theory of Computation \\ 
%  Computer Graphics      & \ \ & Machine Vision      & \ \ & Programming Languages \\
%  Software Engineering   & \ \ & Algorithms          & \ \ & Artificial Intelligence     \\
%  Operating Systems      & \ \ & Databases           & \ \ & Computer Architecture \\
%  Numerical Methods      & \ \ & Graph Theory        & \ \ & Differential Equations      \\
%  Probability Theory     & \ \ & Number Theory       & \ \ & Differential Geometry       \\
%  Advanced Calculus      & \ \ & Abstract Algebra    & \ \ & Advanced Combinatorics   \\
%  \end{tabular}

\section{\textsc{Intereses\\ Personales}}
\employer{}
\title{}
\location{} 
\dates{}
Brazilian Jiu-Jitsu, Judo, Critica de Cine, Tecnologias Web 2.0, Software Libre.
\newline     
\newline


\section{\textsc{Referencias}}

%% Ing. John William Branch Msc,Phd.\\
%% Profesor Asociado \\
%% Escuela de Sistemas \\
%% Facultad de Minas \\
%% Universidad Nacional de Colombia - Sede  Medell\'{i}n\\
%% Tel\'{e}fono: +57 4 4255374 \\
%% E-mail: jwbranch@unalmed.edu.co\\
%%\\  
%% Ing. Andr\'{e}s Osorio Arias Msc,Phd.\\
%% Profesor Asociado \\
%% School of Geological and Environmental Sciences \\
%% Facultad de Minas \\
%% Universidad Nacional de Colombia - Sede Medell\'{i}n\\
%% Tel\'{e}fono: +57 4 4255220 \\
%% E-mail: afosorioar@unal.edu.co\\
%% \\  
 Ing. Esteban Salazar Guzman\\
 Analista de Integraci\'{o}n  \\
 SETI S.A. \\
 Tel\'{e}fono: +57 4 321-5151 \\
 Cell: +57 3003943091 \\ 
 E-mail: estebansalazar@seti.com.co\\
 \\  
 Ing. Jairo Jos\'{e} Espinosa Msc,Phd.\\
 Profesor Asociado \\
 Escuela de Ingenier\'{i}a Elecrica y Mec\'{a}nica \\
 Facultad de Minas \\
 Universidad Nacional de Colombia - Sede  Medell\'{i}n\\
 Tel\'{e}fono: +57 4 4255284 \\
 E-mail: jairo.espinosa@ieee.org\\
 \\  
 Ing. Rafael Ignacio Larios\\
 Analista Desarrollador \\
 Grupo Bancolombia S.A. \\
 Tel\'{e}fono: +57 4 312-9189  \\
 E-mail: rlarios@bancolombia.com.co 

 Ing. Andr\'{e}s Jimenez\\
 Analista Senior \\
 Accenture \\
 Tel\'{e}fono: +57 4 350-9000 ext 9521  \\
 E-mail: andres.f.jimenez@accenture.com 
%% E-mail: esalaza@gmail.com\\
%% \\  
\end{resume}
\end{document}
